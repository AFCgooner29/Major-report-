\section{Problem Formulation}
 Comparing the processing time of graph generation using parrallel as well as the serial python programming. For parrallel processing we will be using the combination of tools like NetworkX, OSMnx and GraphFrames.\\
\noindent When analytical solution is impossible, which was discussed by eg. Alexander Sadovsky. This means that we have to apply numerical methods in order to find the solution. This does not define that we must do calculations with computer although it usually happens so because of the number of required operations.

\section{Facilities required for proposed work}
\subsection{Hardware Requirements}
\begin{itemize}
\item Operating System: Linux
\item Processor Speed: 512KHz or more
\item RAM: Minimum 1GB
\end{itemize}
\subsection{Software Requirements}
\begin{itemize}
\item Softwares: NetworkX, OSMnx, GraphFrame
\item Programming Language: Python 2.7+
\end{itemize}

\section{Methodology}
\begin{itemize}
\item Using OSMnx to get the data from OSM servers. 
\item Using NetworkX to create different graphs using different algorithms.
\item Performing Analysis on graphs.
\item Using Apache Spark via GraphFrames API to generate graphs and analyze them.
\item Using analysis to solve real world problems.
\item Comparing the processing time in both cases on different sizes of datasets.
\end{itemize}

\section{Project Work} 
\textbf{Studied Previous System:}\\
Before starting the project. \\\\
\textbf{Learn the usage of various softwares:}\\
Before starting with project, we have to go through the basics of tools like OSMnx, NetworkX and GraphFrame. We also have to study about the various formats in which data is accepted by the tools. \\\\
\textbf{Get Familiar with Different methods and their algorithms:}\\
Once, we have gone through algorithms of these softwares and tools, the implementation becomes easy.\\\\
\textbf{Input:}\\
Input values are taken from user or default values defined in the file are used.\\\\
\textbf{Output:}\\
The iterations are performed and processing times are computed.

