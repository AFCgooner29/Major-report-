
\section{Feasibility Study}
This study is made to see if the project on completion will serve the purpose of the organization for the amount of work, effort and the time that spend on it. Feasibility study lets the developer foresee the future of the project and the usefulness.\\
A feasibility study of a system proposal is according to its workability, which is the impact on the organization, ability to meet their user needs and effective use of resources. Carrying out a feasibility study involves information assessment, information collection and report writing. The information assessment phase identifies the information that is required to answer the three questions set out above.\\
Once the information has been identified, you should question information sources to discover the answers to these questions Thus when a new application is proposed it normally goes through a feasibility study before it is approved for development.\\\\
A feasibility study is designed to provide an overview of the primary issues related to a business idea. The purpose is to identify any make or break issues that would prevent your business from being successful in the marketplace. In other words, a feasibility study determines whether the business idea makes sense. A thorough feasibility analysis provides a lot of information necessary for the business plan. For example, a good market analysis is necessary in order to determine the project's feasibility. This information provides the basis for the market section of the business plan.\\
The objective of the feasibility study is to establish the reasons for developing the software that is acceptable to users, adaptable to change and conformable to established standards.\\
Objectives of feasibility study are listed below:
\begin{itemize}
	\item To analyze whether the software will meet organizational requirements.
	\item To determine whether the software can be implemented using the current technology and within the specified budget and schedule.
	\item To determine whether the software can be integrated with other existing software.
\end{itemize}

\section{Types of Feasibility}

\subsection{Technical Feasibility}
Technical feasibility is one of the first studies that must be conducted after the project has been selected. The main objective is to make sure all the technical requirements shold be analyzedd and made sure proper technologies are vailable and wisely chosn to make sure the project reaches its desired conclusion. The following should be taken to consideration:
\begin{itemize}
	\item Technologies are searched and Graph analysis tools are available and chosen accordingly.
	\item The Technologies can be implemented with given resources.
	\item The human and economic factor is not a problem.
	\item The problem is to analyze the technologies available and chose wisely.
\end{itemize}

The system must be evaluated from the technical point of view first. The assessment of this feasibility must be based on an outline design of the system requirement in the terms of input, output, programs and procedures. Having identified an outline system, the investigation must go on to suggest the type of equipment, required method developing the system, of running the system once it has been designed. Technical feasibility assesses the current resources (such as hardware and software) and technology, which are required to accomplish user requirements in the software within the allocated time and budget. For this, the software development team ascertains whether the current resources and technology can be upgraded or added in the software to accomplish specified user requirements. A Technical feasibility also performs the following tasks.

\begin{itemize}
	\item Analyzes the technical skills and capabilities of the software development team members, In this case technical skills are available with the team members.
	\item Determines whether the relevant technology is stable and established, In this case technology used is Networkx and GraphFrames.
	\item Ascertains that the technology chosen for software development has a large number of users so that they can be consulted when problems arise or improvements are requisred, much needed support is availble online in this case.
\end{itemize}s

Technical issues raised during the investigation are:
\begin{itemize}
	\item Does the technologies chosen can meet the requirements of task to be fullfiled?, yes the technologies are more than capable.
	\item Can the system expand if developed?, scalabilty is the main feature of GraphFrames Technology.
\end{itemize}

The project should be developed such that the necessary functions and performance are achieved within the constraints. The project is developed within latest technology. Through the technology may become obsolete after some period of time, due to the fact that never version of same software supports older versions, the system may still be used. So there are minimal constraints involved with this project. The system has been developed using PHP the project is technically feasible for development.

\subsection{Economic Feasibility}
The purpose of the economic feasibility assessment is to determine the positive economic benefits to the organization that the proposed system will provide. It includes quantification and identification of all the benefits expected. This assessment typically involves a cost/ benefits analysis.

Economic feasibility is the cost and logistical outlook for a business project or endeavor. Prior to embarking on a new venture, most businesses conduct an economic feasibility study, which is a study that analyzes data to determine whether the cost of the prospective new venture will ultimately be profitable to the company. Economic feasibility is sometimes determined within an organization, while other times companies hire an external company that specializes in conducting economic feasibility studies for them.\\
The developing system must be justified by cost and benefit. Criteria to ensure that effort is concentrated on project, which will give best, return at the earliest. One of the factors, which affect the development of a new system, is the cost it would require. Economic feasibility determines whether the required software is capable of generating financial gains for an organization. In addition, it is necessary to consider the benefits that can be achieved by developing the software. Software is said to be economically feasible if it focuses on the issues listed below.
\begin{itemize}
	\item Cost incurred on software development to produce long-term gains for an organization.
	\item Cost required to conduct full software investigation (such as requirements elicitation and requirements analysis).
	\item Cost of hardware, software, development team, and training.
\end{itemize}

The following are some of the important financial conclusions are made during preliminary investigation:
\begin{itemize}
	\item The costs and economic constraints won't be a problem.
\end{itemize}

Since the system is developed as part of project work, there is no manual cost to spend for the proposed system. 
\noindent Economic analysis is the most frequently used method to determine the cost/benefit factor for evalu-
ating the effectiveness of a new system. In this analysis we determine whether the benefit is gained
according to the cost invested to develop the project or not. If benefits outweigh costs, only then
the decision is made to design and implement the system. It is important to identify cost and benefit
factors, which can be categorized as follows:
\begin{itemize}
\item Development Cost
\item Operation Cost
\end{itemize}
This System is Economically feasible with 0 Development and Operating Charges
as it is developed in Qt Framework and Octave which is open source technology and is available free of cost on the internet.

\subsection{Operational Feasibility}
\noindent Operational feasibility is a measure of how well a project solves the problems, and takes advantage of the opportunities identified during scope definition and how it satisfies the requirements identified in the requirements analysis phase of system development. All the operations performed in the software are very quick and satisfy all the requirements.
\subsection{Technological Feasibility}
\noindent Technological feasibility is carried out to determine whether the project has the capability, in terms
of software, hardware, personnel to handle and fulfill the user requirements. The assessment is based
on an outline design of system requirements in terms of Input, Processes, Output and Procedures.
Automated Building Drawings is technically feasible as it is built up using various open source technologies and it can run on any platform.
\subsection{Behavioral Feasibility}
Behavioral feasibility assesses the extent to which the required software performs a series of steps to solve business problems and user requirements. It is a measure of how well the solution of problems or a specific alternative solution will work in the organization. It is also measure of how people feel about the system. If the system is not easy to operate, than operational process would be difficult. The operator of the system should be given proper training. The system should be made such that the user can interface the system without any problem.

Operational feasibility is a measure of how well a proposed system solves the problems, and takes advantage of the opportunities identified during scope definition and how it satisfies the requirements identified in the requirements analysis phase of system development. The operational feasibility assessment focuses on the degree to which the proposed development projects fits in with the existing business environment and objectives with regard to development schedule, delivery date, corporate culture, and existing business processes.

To ensure success, desired operational outcomes must be imparted during design and development. These include such design-dependent parameters such as reliability, maintainability, supportability, usability, producibility, disposability, sustainability, affordability and others. These parameters are required to be considered at the early stages of design if desired operational behaviors are to be realized. A system design and development requires appropriate and timely application of engineering and management efforts to meet the previously mentioned parameters. A system may serve its intended purpose most effectively when its technical and operating characteristics are engineered into the design. Therefore, operational feasibility is a critical aspect of systems engineering that needs to be an integral part of the early design phasesThis feasibility is dependent on human resources (software development team) and involves visualizing whether the software will operate after it is developed and be operative once it is installed. Operational feasibility also performs the following tasks.

\begin{itemize}
	\item Determines whether the problems anticipated in user requirements are of high priority.
	\item Determines whether the solution suggested by the software development team is acceptable.
	\item Analyzes whether users will adapt to a new software.
	\item Determines whether the organization is satisfied by the alternative solutions proposed by the software development team.
\end{itemize}

Following conclusions are made after analysis:
\begin{itemize}
	\item Enough support is available for GraphFrames and NetworkX.
	\item No harm is caused in development of any kind.
	\item The project would be beneficial because it satisfies the objectives when developed and installed. All behavioral aspects are considered carefully and conclude that the project is behaviorally feasible.
\end{itemize}

\section{Objective of Project}
 	The main objective of the project is to compare the processing time of graph creation using both parrallel and serial processing systems. Subobjectives of the project are: 
\begin{enumerate}
\item Using OSMnx to get the data from OSM servers. 
\item Using NetworkX to create different graphs using different algorithms.
\item Using centrality degree algorithm to plot the graph.
\item Using simple python program to create graph.
\item Comparing the processing time in both cases on different sizes of datasets.
\end{enumerate}



