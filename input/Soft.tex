\section{Software Requirement Analysis}
Software requirement analysis is a process of gathering and interpreting facts, diagnosing problems and the information to recommend improvements on the system. It is a problem solving activity that requires intensive communication between the system users and system developers. System analysis or study is an important phase of any system development process. The system is studied to the minutest detail and analyzed. The system analyst plays the role of the interrogator and dwells deep into the working of the present system. The system is viewed as a whole and the input to the system are identified. The outputs from the organizations are traced to the various processes. System analysis is concerned with becoming aware of the problem, identifying the relevant and decisional variables, analyzing and synthesizing the various factors and determining an optimal or at least a satisfactory solution or program of action.\\

\noindent A detailed study of the process must be made by various techniques like interviews, questionnaires etc. The data collected by these sources must be scrutinized to arrive to a conclusion. The conclusion is an understanding of how the system functions. This system is called the existing system. Now the existing system is subjected to close study and problem areas are identified. The designer now functions as a problem solver and tries to sort out the difficulties that the enterprise faces. The solutions are given as proposals. The proposal is then weighed with the existing system analytically and the best one is selected. The proposal is presented to the user for an endorsement by the user. The proposal is reviewed on user request and suitable changes are made. This is loop that ends as soon as the user is satisfied with proposal.\\

\noindent Preliminary study is the process of gathering and interpreting facts, using the information for further studies on the system. Preliminary study is problem solving activity that requires intensive communication between the system users and system developers. It does various feasibility studies. In these studies a rough figure of the system activities can be obtained, from which the decision about the strategies to be followed for effective system study and analysis can be taken.
\subsection{Functional Requirements}
\begin{itemize}
\item {\bf Specific Requirements}: This phase covers the whole requirements 
for the system. After understanding the system we need the input data 
to the system then we watch the output and determine whether the output 
from the system is according to our requirements or not. So what we have 
to input and then what we'll get as output is given in this phase. This 
phase also describe the software and non-function requirements of the 
system.
\item {\bf Input Requirements of the System}
\begin{enumerate} 
\item Data set of an area.
\item Type of graph to be generated.
\item Whether to use parrallel or serial processing.
\end{enumerate}
\vskip 0.5cm
\item {\bf Output Requirements of the System}
\begin{enumerate} 
\item Final time computation after processes
\item Results from the analysis queries. 
\end{enumerate}
\vskip 0.5cm
\item {\bf Software Requirements}
\begin{enumerate} 
\item Programming language: Python 2.7+
\item software: \LaTeX{}
\item Processing Tchnology: GraphFrame or NetworkX 
\item Text Editor: Vim
\item Operating System: Ubuntu 14.04+
\item Revision System: Git
\end{enumerate}
\end{itemize}

\section{Other Specifications}

A Software Requirements Analysis for a software system is a complete 
description of the behavior of a system to be developed. It include functional Requirements
and Software Requirements. In addition to these, the SRS contains 
non-functional requirements. Non-functional requirements are 
requirements which impose constraints on the design or implementation.
\begin{itemize}
\item{\bf Purpose}: To compare the processing time of serial and parrallel processing using graph generation of large data set.
Perform most of difficult Calculation work.
\end{itemize}


