\section{Overview}
\subsection{What is OSM?}
OpenStreetMap (OSM) is a collaborative project to create a free editable map of the world. The creation and growth of OSM has been motivated by restrictions on use or availability of map information across much of the world, and the advent of inexpensive portable satellite navigation devices. OSM is considered a prominent example of volunteered geographic information.\\

Created by Steve Coast in the UK in 2004, it was inspired by the success of Wikipedia and the predominance of proprietary map data in the UK and elsewhere. Since then, it has grown to over 2 million registered users, who can collect data using manual survey, GPS devices, aerial photography, and other free sources. This crowdsourced data is then made available under the Open Database Licence. The site is supported by the OpenStreetMap Foundation, a non-profit organisation registered in England and Wales.\\

Rather than the map itself, the data generated by the OpenStreetMap project is considered its primary output. The data is then available for use in both traditional applications, like its usage by Craigslist, OsmAnd, Geocaching, MapQuest Open, JMP statistical software, and Foursquare to replace Google Maps, and more unusual roles like replacing the default data included with GPS receivers. OpenStreetMap data has been favourably compared with proprietary datasources, though data quality varies worldwide.

\subsection{ Map production}
Map data is collected from scratch by volunteers performing systematic ground surveys using tools such as a handheld GPS unit, a notebook, digital camera, or a voice recorder. The data is then entered into the OpenStreetMap database. Mapathon competition events are also held by OpenStreetMap team and by non-profit organisations and local governments to map a particular area.\\

The availability of aerial photography and other data from commercial and government sources has added important sources of data for manual editing and automated imports. Special processes are in place to handle automated imports and avoid legal and technical problems.
\subsection{ Route planning}

In February 2015, OpenStreetMap added route planning functionality to the map on its official website. The routing uses external services, namely OSRM, GraphHopper and MapQuest.\\

There are other routing providers and applications listed in the official Routing wiki.

\subsection{Data storage}
The OSM data primitives are stored and processed in different formats.\\

The main copy of the OSM data is stored in OSM's main database. The main database is a PostgreSQL database with PostGIS extension, which has one table for each data primitive, with individual objects stored as rows. All edits happen in this database, and all other formats are created from it.\\

For data transfer, several database dumps are created, which are available for download. The complete dump is called planet.osm. These dumps exist in two formats, one using XML and one using the Protocol Buffer Binary Format (PBF).\\

The LinkedGeoData data uses the GeoSPARQL and well-known text (WKT) RDF vocabularies to represent OpenStreetMap data. It is a work of the Agile Knowledge Engineering and Semantic Web (AKSW) research group at the University of Leipzig, a group mostly known for DBpedia.

\section{User Requirement Analysis}
\begin{enumerate}
	\item Using OSMnX to get the data from OSM servers.
	\item Using networkX to plot the graph with parrallel processing.
	\item Using simple python program to create the graph. 
	\item Create the road network of an area.
	\item Compare the processing times of the parrallel vs serial processors. 
\end{enumerate}